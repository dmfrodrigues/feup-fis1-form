\documentclass{form}
\author{Diogo Rodrigues (dmfrodrigues2000@gmail.com)}
\title{FIS1 -- Equations form}

\newcolumntype{P}[1]{>{\centering\arraybackslash}p{#1}}
% Document
\begin{document}
\section*{Kinematics}
\begin{center} \begin{tabular}{c | c | c | p{115mm}}
    $\displaystyle \vec{r} = \vec{r}_0 + \int_0^t{\vec{v}\,dt}$ &
    $\displaystyle \vec{v} = \vec{v}_0 + \int_0^t{\vec{a}\,dt}$ &
    $\displaystyle a_x = v_x \frac{dv_x}{dx}$ &
    Consider a body of mass $m$, to which a constant force of magnitude $F=m\,a$ is applied. Then it holds that $v_f^2 - v_i^2 = 2\,a\,\Delta x$, or $v^2 = 2\,a\,\Delta x$.
\end{tabular} \end{center}

\begin{center}
    \setlength\extrarowheight{9pt}
    \begin{tabular}{p{50mm} | P{44mm} | P{44mm} | P{44mm}}
        \textbf{Relative motion} & $\vec{r}' = \vec{r} + \vec{r}_O'$ & $\vec{v}' = \vec{v} + \vec{v}_O'$ & $\vec{a}' = \vec{a} + \vec{a}_O'$ \\[7pt] \hline
        \textbf{Curvilinear motion} & $\vec{v} = \dot{s} \unitvec{e}_t$ & \multicolumn{2}{c}{$\displaystyle \vec{a} = \vec{a}_t + \vec{a}_n = \dot{v} \unitvec{e}_t+\frac{v^2}{R}\unitvec{e}_n$} \\[7pt] \hline
        \textbf{Circular motion} & $\displaystyle \alpha = \omega \frac{d\omega}{d\theta}$ & \multicolumn{2}{c}{$\vec{a} = \vec{a}_t + \vec{a}_n = \vec{\alpha} \times \vec{r} + \vec{\omega} \times \vec{v}$} \\[7pt]
    \end{tabular}
\end{center}
\section*{Mechanics}
\begin{center} \begin{tabular*}{\textwidth}{@{\extracolsep{\fill}}ccccccc}
    $\displaystyle \vec{F}_R = \sum_{i=1}^{N}{\vec{F}_i}$ &
    $\displaystyle \vec{F} = \frac{d\vec{p}}{dt}$ &
    $\displaystyle \vec{p} = m\vec{v}$ & 
    $\displaystyle \vec{F}_R = m \vec{a}_{CM}$ &
    $\displaystyle \vec{F}_g = m\vec{g}$ & 
    \begin{minipage}[c]{30mm}
        \centering
        \textbf{Static friction} \\
        $F_e \leq \mu_e N$
    \end{minipage} &
    \begin{minipage}[c]{30mm}
        \centering
        \textbf{Kinetic friction} \\
        $F_c \leq \mu_c N$
    \end{minipage}
\end{tabular*} \end{center}
\begin{center} \begin{tabular*}{\textwidth}{|@{\extracolsep{\fill}}lccc|} \hline
    \begin{minipage}[c]{50mm}
        \textbf{Sphere in a fluid} \\
        $N_R$ -- Reynolds number
    \end{minipage} &
    $\displaystyle N_R = rv\left(\frac{\rho}{\eta}\right)$ &
    $F_f = 6\pi \eta r v$ ($N_R < 1$) & 
    $\displaystyle F_f = \frac{\pi}{4} \rho r^2 v^2$ ($N_R > 10^3$) \\ \hline
\end{tabular*} \end{center}

\section*{Rigid body dynamics}
\begin{center} \begin{tabular*}{\textwidth}{@{\extracolsep{\fill}}ccccc}
    $\displaystyle \sum_{i=1}^N{\vec{F}_{int}} = \vec{0}$ &
    $\displaystyle \vec{r}_{CM} = \frac{1}{m} \iiint_Q{\vec{r}\,dm}$ &
    $\displaystyle \vec{\tau} = \vec{r} \times \vec{F}$ &
    $\displaystyle \sum_{i=1}^N{\vec{\tau}_i} = \vec{\tau}_R = \matr{I} \vec{\alpha}$ &
    $\displaystyle I_{zz} = \iiint_Q{\vec{r}^2\,dm} = \iiint_Q{(x^2+y^2)\,dm}$
\end{tabular*} \end{center}
\begin{center} \begin{tabular}{|l c|} \hline
    \begin{minipage}[c]{167mm}
        \textbf{Parallel axis theorem} \\
        $I_{CM}$ -- Inertia at the CM axis | $I$ -- Inertia at the axis parallel to the CM axis | $d$ -- Distance between axis
    \end{minipage} & 
    $I = I_{CM} + m d^2$ \\ \hline
\end{tabular} \end{center}
\subsection*{Moments of inertia of some solids}
\begin{center}
    \setlength\extrarowheight{16pt}
    \begin{tabular}{p{67mm} c | p{67mm} c}
        \textbf{Point particle} & $I = mr^2$ &
        \begin{minipage}[c]{55mm}
            \textbf{Spherical shell} \\
            Internal radius $r$, external radius $R$
        \end{minipage} & $\displaystyle I = \frac{2}{5} m \left( \frac{R^5-r^5}{R^3-r^3} \right)$ \\[14pt] \hline
        \begin{minipage}[c]{60mm}
            \textbf{Thin cylinder} \\
            Rotation around an axis perpendicular to the cylinder's axis
        \end{minipage} & $\displaystyle I = \frac{1}{12} m L^2$ &
        \begin{minipage}[c]{60mm}
            \textbf{Cylindrical tube with thick walls} \\
            Rotation around the cylinder's axis \\
            Internal radius $r$, external radius $R$
        \end{minipage} & $\displaystyle I = \frac{1}{2}m(R^2+r^2)$ \\[14pt]
    \end{tabular}
\end{center}

\section*{Work and energy}
\begin{center} \begin{tabular*}{\textwidth}{@{\extracolsep{\fill}}ccccc}
    \multirow{2}{*}{$\displaystyle W_{12} = \int_{s_1}^{s_2}{F_t \, ds} = \int_{\vec{r}_1}^{\vec{r}_2}{\vec{F} \cdot d\vec{r}}$} &
    \multirow{2}{*}{$\displaystyle E_k = \frac{1}{2}mv_{CM}^2 + \frac{1}{2}I_{CM}\omega^2$} &
    $E_m = E_k + U$ & 
    \multirow{2}{*}{$\displaystyle U = - \int_{\vec{r}_0}^{\vec{r}}{\vec{F} \cdot d\vec{r}}$} &
    \multirow{2}{*}{$\displaystyle \int_{s_1}^{s_2}{F_t^{nc} \, ds = \Delta E_m}$} \\
    & & $U_g = mgz$ & &
\end{tabular*} \end{center}

\subsection*{Harmonic oscillator}
\begin{center} \begin{tabular*}{\textwidth}{@{\extracolsep{\fill}} cccc}
    $\Omega = \sqrt{k/m} = 2\pi f$ &
    $s = A \sin{(\Omega t + \varphi_0)}$ &
    $U_e = \frac{1}{2} k s^2$ & 
    $E_m = \frac{1}{2} m v^2 + \frac{1}{2} k s^2$
\end{tabular*} \end{center}

\section*{Mathematics}
\begin{center} \begin{tabular}{P{70mm} | P{47mm} | P{70mm}}
    \begin{minipage}[c]{0.30\textwidth}
        \centering
        $\vec{a} \cdot \vec{b} = a\,b\,\cos{\theta} = a_x b_x + a_y b_y + a_z b_z$ \\[5pt]
        $a = \sqrt{\vec{a} \cdot \vec{a}}$
    \end{minipage} &
    $\vec{a} \times \vec{b} = -\vec{b} \times \vec{a}$ &
    $\vec{a} \times \vec{b} = a\,b\,\sin{\theta}\,\unitvec{n} =
    \begin{vmatrix}
        \unitvec{i} & \unitvec{j} & \unitvec{k} \\ 
        a_x & a_y & a_z \\ 
        b_x & b_y & b_z
    \end{vmatrix}$
\end{tabular} \end{center}

\section*{Maxima}
\begin{center} \begin{tabular}{l p{52mm} | l p{52mm}}
    \texttt{? log;} & Help about a function & \texttt{remvalue(a);} & Delete \texttt{a} from memory \\ \hline
    \texttt{a: makelist(i\^{}3,i,2,6,0.6);} & Create list from expression \texttt{i\^{}3}, index \texttt{i}, from \texttt{2} to \texttt{6} with step \texttt{0.6} & \texttt{solve(\%,x);} & Solve equation with unknown \texttt{x} \\ \hline
    \texttt{allroots(\%);} & Find all roots of the polynomial & & 
\end{tabular} \end{center}
\newpage
\section*{Space phase / Systems}

TODO

\subsection*{Linear systems}
\begin{center} \begin{tabular}{P{62mm} P{62mm} P{62mm}}
    $\displaystyle \vec{r} = \begin{bmatrix}
        x_1 \\
        x_2
    \end{bmatrix}$ &
    $\displaystyle \frac{d\vec{r}}{dt} = \matr{J}\vec{r}$ &
    $\displaystyle \matr{J} = \begin{bmatrix}
        \partial f_1 & \partial x_1 & \partial f_1 & \partial x_2 \\
        \partial f_2 & \partial x_1 & \partial f_2 & \partial x_2 
    \end{bmatrix}$
\end{tabular} \end{center}
\begin{center} \begin{tabular}{p{75mm} p{115mm}}
    \textbf{Eigenvalues:} $\lambda^2 - \tr{\matr{A}} + |A| = 0$
    &
    \texttt{eigenvectors(A);} \texttt{[1][1]} - eigenvalues; \texttt{[1][2]} - eigenvalues multiplicity; \texttt{[2]} - eigenvectors;
\end{tabular} \end{center}
\begin{center} \begin{tabular}{p{65mm} p{65mm} p{55mm}} \hline \hline
    \textbf{Eigenvalues $\lambda$} & \textbf{Type of point} & \textbf{Stability} \\ \hline
    2 reals, opposite signals & saddle point & unstable \\
    2 reals, positive & repulsive node & unstable \\
    2 reals, negative & attractive node & stable \\
    2 complex; positive real part & repulsive focus & unstable \\
    2 complex; negative real part & attractive focus & stable \\
    2 pure imaginaries & center & stable \\
    1 real, positive & improper repulsive node & unstable \\
    1 real, negative & improper attractive node & unstable \\ \hline \hline
\end{tabular} \end{center}
\end{document}
